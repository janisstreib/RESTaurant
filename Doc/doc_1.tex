\documentclass{beamer}
\usepackage[utf8]{inputenc}
\usepackage{hyperref}
\usepackage{forloop}
\usepackage{amsmath,amsfonts,amssymb}
\useoutertheme {smoothbars}

\usepackage{listings}
\usepackage{color}
\definecolor{name}{rgb}{0.5,0.5,0.5}
\definecolor{javared}{rgb}{0.6,0,0} % for strings
\definecolor{javagreen}{rgb}{0.25,0.5,0.35} % comments
\definecolor{javapurple}{rgb}{0.5,0,0.35} % keywords
\definecolor{javadocblue}{rgb}{0.25,0.35,0.75} % javadoc
\lstset{language=Java,
	basicstyle=\ttfamily,
	keywordstyle=\color{javapurple}\bfseries,
	stringstyle=\color{javared},
	commentstyle=\color{javagreen},
	morecomment=[s][\color{javadocblue}]{/**}{*/},
	numbers=left,
	numberstyle=\tiny\color{black},
	stepnumber=2,
	numbersep=10pt,
	tabsize=4,
	showspaces=false,
	showstringspaces=false}
\defbeamertemplate*{footline}{smoothbars theme}
{%
	\begin{beamercolorbox}[colsep=1.5pt]{upper separation line foot}
	\end{beamercolorbox}
	\begin{beamercolorbox}[ht=6ex,dp=1.125ex,%
		leftskip=.3cm,rightskip=.3cm plus1fil]{title in head/foot}%
		\leavevmode{\usebeamerfont{title in head/foot}\insertshorttitle}%
		\hfill%
		{\usebeamerfont{author in head/foot}\usebeamercolor[fg]{author in head/foot}\insertshortauthor}
		\vspace{.8mm}
		
		\url{http://creativecommons.org/licenses/by-nd/4.0/}%
	\end{beamercolorbox}%
	\begin{beamercolorbox}[colsep=1.5pt]{lower separation line foot}
	\end{beamercolorbox}
}
\author[Janis Streib]{Janis Streib \\ CC-BY-ND-4.0}
\newcounter{ct}

\begin{document}
%\def\pause{}
\title{RESTaurant for Android: Programmierung einer Sitzplatzreservierung \\ Teil I}   

\date{09.-13.07.2018} 

\frame{\titlepage} 

\frame{\frametitle{Inhalt}\tableofcontents} 

\section{JDK installieren}
\subsection{JDK}
\frame{\frametitle{JDK installieren}
 \begin{itemize}
     \item Java Development Kit (JDK), der neben der JVM (Java Virtual Machine) auch Compiler und Java Systembibliotheken enthält (aktuell: JDK10): \\
 	\url{http://www.oracle.com/technetwork/java/javase/downloads/jdk10-downloads-4416644.html}
\end{itemize}
}
\section{Android Studio installieren}
\subsection{Android Studio}
\frame{\frametitle{Android Studio installieren}
	\begin{itemize}
		\item Zur Gestaltung der späteren Benutzeroberfläche der App benötigen wir die offizielle Android-Entwicklungsumgebung
		\item Android Studio basiert auf der populären IDE IntelliJ
		\item Android Studio: \\
		\url{https://developer.android.com/studio/}
	\end{itemize}
}
\forloop{ct}{1}{\value{ct} < 10}%
{%
	\frame{\frametitle{Android Studio einrichten}		
			\includegraphics[width=116mm]{img/inst\arabic{ct}}	
	}
}

\section{AVDs}
	\frame{\frametitle{AVDs}		
		\begin{itemize}
			\item Das Android-SDK stellt die Möglichkeit eines Android-Emulators zur Verfügung
			\item Ermöglicht die Ausführung beliebiger Android-Versionen und Formfaktoren ohne echtes Smartphone/Tablet/Watch/...
			\item AVD := Android Virtual Device
		\end{itemize}
	}
\section{NetBeans IDE installieren}
\subsection{NetBeans}
\frame{\frametitle{NetBeans IDE installieren}
	\begin{itemize}
		\item NetBeans IDE (for Java SE): \\
		\url{https://netbeans.org/}
	\end{itemize}
}

\end{document}

