\documentclass{beamer}
\usepackage[utf8]{inputenc}
\usepackage{hyperref}
\usepackage{forloop}
\usepackage{amsmath,amsfonts,amssymb}
\useoutertheme {smoothbars}

\defbeamertemplate*{footline}{smoothbars theme}
{%
	\begin{beamercolorbox}[colsep=1.5pt]{upper separation line foot}
	\end{beamercolorbox}
	\begin{beamercolorbox}[ht=2.5ex,dp=1.125ex,%
		leftskip=.3cm,rightskip=.3cm plus1fil]{title in head/foot}%
		\leavevmode{\usebeamerfont{title in head/foot}\insertshorttitle}%
		\hfill%
		{\usebeamerfont{author in head/foot}\usebeamercolor[fg]{author in head/foot}\insertshortauthor}%
	\end{beamercolorbox}%
	\begin{beamercolorbox}[colsep=1.5pt]{lower separation line foot}
	\end{beamercolorbox}
}

\begin{document}

\title{RESTaurant for Android: Programmierung einer Sitzplatzreservierung \\ Teil I}   
\author{Janis Streib} 
\date{02.-06.11.2015} 

\frame{\titlepage} 

\frame{\frametitle{Inhalt}\tableofcontents} 


\section{Teil I: IDE einrichten} 
\subsection{JDK installieren}
\frame{\frametitle{JDK installieren}
 \begin{itemize}
 	\item Java Development Kit (JDK), der Compiler und Java Systembibliotheken enthält (aktuell: JDK8): \\
 	\url{https://www.oracle.com/technetwork/java/javase/downloads/jdk8-downloads-2133151.html}
\end{itemize}
}
\subsection{Android Studio installieren}
\frame{\frametitle{Android Studio installieren}
	\begin{itemize}
		\item Zur gestaltung der späteren Benutzeroberfläche der App benötigen wir die offizielle Android-Entwicklungsumgebung
		\item Android Studio basiert auf der populären IDE IntelliJ
		\item Android Studio: \\
		\url{https://developer.android.com/sdk/index.html}
	\end{itemize}
}
\newcounter{ct}
\forloop{ct}{1}{\value{ct} < 9}%
{%
	\frame{\frametitle{Android Studio einrichten}		
			\includegraphics[width=116mm]{img/inst\arabic{ct}}	
	}
}

\subsection{NetBeans IDE installieren}
\frame{\frametitle{NetBeans IDE installieren}
	\begin{itemize}
		\item NetBeans IDE (for Java SE): \\
		\url{https://netbeans.org/}
		\item NBAndroid plugin zur Androidentwicklung in NetBeans: \\
		\url{https://bitbucket.org/nbandroid/nbandroid/wiki/Installation}
		\item NetBeans Updatesite:
		\url{http://nbandroid.org/updates/updates.xml}
	\end{itemize}
}

\forloop{ct}{1}{\value{ct} < 11}%
{%
\frame{\frametitle{NetBeans IDE einrichten}		
		\includegraphics[width=116mm]{img/net\arabic{ct}}
}
}
\end{document}

